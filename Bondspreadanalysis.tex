% Options for packages loaded elsewhere
% Options for packages loaded elsewhere
\PassOptionsToPackage{unicode}{hyperref}
\PassOptionsToPackage{hyphens}{url}
\PassOptionsToPackage{dvipsnames,svgnames,x11names}{xcolor}
%
\documentclass[
  letterpaper,
  DIV=11,
  numbers=noendperiod]{scrartcl}
\usepackage{xcolor}
\usepackage{amsmath,amssymb}
\setcounter{secnumdepth}{5}
\usepackage{iftex}
\ifPDFTeX
  \usepackage[T1]{fontenc}
  \usepackage[utf8]{inputenc}
  \usepackage{textcomp} % provide euro and other symbols
\else % if luatex or xetex
  \usepackage{unicode-math} % this also loads fontspec
  \defaultfontfeatures{Scale=MatchLowercase}
  \defaultfontfeatures[\rmfamily]{Ligatures=TeX,Scale=1}
\fi
\usepackage{lmodern}
\ifPDFTeX\else
  % xetex/luatex font selection
\fi
% Use upquote if available, for straight quotes in verbatim environments
\IfFileExists{upquote.sty}{\usepackage{upquote}}{}
\IfFileExists{microtype.sty}{% use microtype if available
  \usepackage[]{microtype}
  \UseMicrotypeSet[protrusion]{basicmath} % disable protrusion for tt fonts
}{}
\makeatletter
\@ifundefined{KOMAClassName}{% if non-KOMA class
  \IfFileExists{parskip.sty}{%
    \usepackage{parskip}
  }{% else
    \setlength{\parindent}{0pt}
    \setlength{\parskip}{6pt plus 2pt minus 1pt}}
}{% if KOMA class
  \KOMAoptions{parskip=half}}
\makeatother
% Make \paragraph and \subparagraph free-standing
\makeatletter
\ifx\paragraph\undefined\else
  \let\oldparagraph\paragraph
  \renewcommand{\paragraph}{
    \@ifstar
      \xxxParagraphStar
      \xxxParagraphNoStar
  }
  \newcommand{\xxxParagraphStar}[1]{\oldparagraph*{#1}\mbox{}}
  \newcommand{\xxxParagraphNoStar}[1]{\oldparagraph{#1}\mbox{}}
\fi
\ifx\subparagraph\undefined\else
  \let\oldsubparagraph\subparagraph
  \renewcommand{\subparagraph}{
    \@ifstar
      \xxxSubParagraphStar
      \xxxSubParagraphNoStar
  }
  \newcommand{\xxxSubParagraphStar}[1]{\oldsubparagraph*{#1}\mbox{}}
  \newcommand{\xxxSubParagraphNoStar}[1]{\oldsubparagraph{#1}\mbox{}}
\fi
\makeatother


\usepackage{longtable,booktabs,array}
\usepackage{calc} % for calculating minipage widths
% Correct order of tables after \paragraph or \subparagraph
\usepackage{etoolbox}
\makeatletter
\patchcmd\longtable{\par}{\if@noskipsec\mbox{}\fi\par}{}{}
\makeatother
% Allow footnotes in longtable head/foot
\IfFileExists{footnotehyper.sty}{\usepackage{footnotehyper}}{\usepackage{footnote}}
\makesavenoteenv{longtable}
\usepackage{graphicx}
\makeatletter
\newsavebox\pandoc@box
\newcommand*\pandocbounded[1]{% scales image to fit in text height/width
  \sbox\pandoc@box{#1}%
  \Gscale@div\@tempa{\textheight}{\dimexpr\ht\pandoc@box+\dp\pandoc@box\relax}%
  \Gscale@div\@tempb{\linewidth}{\wd\pandoc@box}%
  \ifdim\@tempb\p@<\@tempa\p@\let\@tempa\@tempb\fi% select the smaller of both
  \ifdim\@tempa\p@<\p@\scalebox{\@tempa}{\usebox\pandoc@box}%
  \else\usebox{\pandoc@box}%
  \fi%
}
% Set default figure placement to htbp
\def\fps@figure{htbp}
\makeatother


% definitions for citeproc citations
\NewDocumentCommand\citeproctext{}{}
\NewDocumentCommand\citeproc{mm}{%
  \begingroup\def\citeproctext{#2}\cite{#1}\endgroup}
\makeatletter
 % allow citations to break across lines
 \let\@cite@ofmt\@firstofone
 % avoid brackets around text for \cite:
 \def\@biblabel#1{}
 \def\@cite#1#2{{#1\if@tempswa , #2\fi}}
\makeatother
\newlength{\cslhangindent}
\setlength{\cslhangindent}{1.5em}
\newlength{\csllabelwidth}
\setlength{\csllabelwidth}{3em}
\newenvironment{CSLReferences}[2] % #1 hanging-indent, #2 entry-spacing
 {\begin{list}{}{%
  \setlength{\itemindent}{0pt}
  \setlength{\leftmargin}{0pt}
  \setlength{\parsep}{0pt}
  % turn on hanging indent if param 1 is 1
  \ifodd #1
   \setlength{\leftmargin}{\cslhangindent}
   \setlength{\itemindent}{-1\cslhangindent}
  \fi
  % set entry spacing
  \setlength{\itemsep}{#2\baselineskip}}}
 {\end{list}}
\usepackage{calc}
\newcommand{\CSLBlock}[1]{\hfill\break\parbox[t]{\linewidth}{\strut\ignorespaces#1\strut}}
\newcommand{\CSLLeftMargin}[1]{\parbox[t]{\csllabelwidth}{\strut#1\strut}}
\newcommand{\CSLRightInline}[1]{\parbox[t]{\linewidth - \csllabelwidth}{\strut#1\strut}}
\newcommand{\CSLIndent}[1]{\hspace{\cslhangindent}#1}



\setlength{\emergencystretch}{3em} % prevent overfull lines

\providecommand{\tightlist}{%
  \setlength{\itemsep}{0pt}\setlength{\parskip}{0pt}}



 


\KOMAoption{captions}{tableheading}
\makeatletter
\@ifpackageloaded{caption}{}{\usepackage{caption}}
\AtBeginDocument{%
\ifdefined\contentsname
  \renewcommand*\contentsname{Table of contents}
\else
  \newcommand\contentsname{Table of contents}
\fi
\ifdefined\listfigurename
  \renewcommand*\listfigurename{List of Figures}
\else
  \newcommand\listfigurename{List of Figures}
\fi
\ifdefined\listtablename
  \renewcommand*\listtablename{List of Tables}
\else
  \newcommand\listtablename{List of Tables}
\fi
\ifdefined\figurename
  \renewcommand*\figurename{Figure}
\else
  \newcommand\figurename{Figure}
\fi
\ifdefined\tablename
  \renewcommand*\tablename{Table}
\else
  \newcommand\tablename{Table}
\fi
}
\@ifpackageloaded{float}{}{\usepackage{float}}
\floatstyle{ruled}
\@ifundefined{c@chapter}{\newfloat{codelisting}{h}{lop}}{\newfloat{codelisting}{h}{lop}[chapter]}
\floatname{codelisting}{Listing}
\newcommand*\listoflistings{\listof{codelisting}{List of Listings}}
\makeatother
\makeatletter
\makeatother
\makeatletter
\@ifpackageloaded{caption}{}{\usepackage{caption}}
\@ifpackageloaded{subcaption}{}{\usepackage{subcaption}}
\makeatother
\usepackage{bookmark}
\IfFileExists{xurl.sty}{\usepackage{xurl}}{} % add URL line breaks if available
\urlstyle{same}
\hypersetup{
  pdftitle={On The Micro-Determinants of Credit Spreads},
  pdfauthor={Nipun Jaiswal and Ana Elisa Lopez-Miranda; (1008726826) and (1008819879)},
  colorlinks=true,
  linkcolor={blue},
  filecolor={Maroon},
  citecolor={Blue},
  urlcolor={Blue},
  pdfcreator={LaTeX via pandoc}}


\title{On The Micro-Determinants of Credit Spreads}
\usepackage{etoolbox}
\makeatletter
\providecommand{\subtitle}[1]{% add subtitle to \maketitle
  \apptocmd{\@title}{\par {\large #1 \par}}{}{}
}
\makeatother
\subtitle{ECO375: Empirical Project}
\author{Nipun Jaiswal and Ana Elisa Lopez-Miranda \and (1008726826) and
(1008819879)}
\date{November 27, 2025}
\begin{document}
\maketitle


\newpage

\section{Introduction}\label{sec-intro}

Finding the micro-determinants for credit bonds is useful for finding
what affects yields over time.

This paper is structured as follows: Section~\ref{sec-lit} is the
literature review situating the research question, Section~\ref{sec-des}
contains the preliminary analyses, Section~\ref{sec-met} contains the
methodology, Section~\ref{sec-res} contains the results, and
Section~\ref{sec-hyp} has the hypothesis testing.

\section{Firm value and capital structure
ratios}\label{firm-value-and-capital-structure-ratios}

Enterprise Value = Market Cap + Total Debt - Cash Enterprise Value
toSales measures how expensive a company is relative to its revnue High
EV/Sales -\textgreater{} market expects high growth OR company is
overvalued High EV -\textgreater{} lower PERCEIVED risk -\textgreater{}
lower spreads

PPE is actual long-term assets (machines, buildings, equipment) High
EV/PPE -\textgreater{} growth-oriented High EV/PPE -\textgreater{}
riskier -\textgreater{} higher spreads

High P/S -\textgreater{} confidence -\textgreater{} lower spreads

Book Value = Assets - Liabilities P/B \textless1 -\textgreater{}
undervaluation -\textgreater{} financial distress -\textgreater{} higher
spreads

Dividend yield -\textgreater{} higher MAY indicate risk

High RATE -\textgreater{} more efficient -\textgreater{} lower spreads

\section{profitability and operational
efficiency}\label{profitability-and-operational-efficiency}

operating margin = earnings before interest and taxes/revenue high
margin -\textgreater{} lower default risk (firm is efficient)
-\textgreater{} lower spreads

High total capital expenses to total assets -\textgreater{} lower
spreads or less cash :\\

low book value per share -\textgreater{} maybe bankrupt

High asset turnover -\textgreater{} efficient -\textgreater{} lower
spreads

\section{Liquidity and Cash flow
ratios}\label{liquidity-and-cash-flow-ratios}

current ratio = current assets (cash + receivables + inventory)/ current
liabilities (obligations due) low current ratio -\textgreater{}
liquidity risk -\textgreater{} higher spreads

Cash dividend coverage ratio = operating cash flow/cash dividends paid
lower coverage -\textgreater{} weak cash flow -\textgreater{} higher
spreads

\section{Leverage and Capital
Structure}\label{leverage-and-capital-structure}

high total debt/shareholder equity -\textgreater{} in debt
-\textgreater{} higher spreads

high total debt/(total debt + equity) -\textgreater{} high reliance on
debt -\textgreater{} high spreads

\section{Bond specific}\label{bond-specific}

Higher amount outstanding -\textgreater{} lower spreads (more liquid?)
smaller amount outstanding -\textgreater{} higher spreads

years to maturity -\textgreater{} more interest rate and credit risk
-\textgreater{} higher spread (risk premium)

low couponds -\textgreater{} high price sensitity? MAYBE. Higher does
not not imply higher here

difference in percentages: basis points

\subsection{Description of variables}\label{description-of-variables}

The variables are as follows

\begin{longtable}[]{@{}
  >{\raggedright\arraybackslash}p{(\linewidth - 2\tabcolsep) * \real{0.4091}}
  >{\raggedright\arraybackslash}p{(\linewidth - 2\tabcolsep) * \real{0.5909}}@{}}
\toprule\noalign{}
\begin{minipage}[b]{\linewidth}\raggedright
Variable
\end{minipage} & \begin{minipage}[b]{\linewidth}\raggedright
Description
\end{minipage} \\
\midrule\noalign{}
\endhead
\bottomrule\noalign{}
\endlastfoot
Issuer & Name of the company that issued the bond \\
Spread & The difference between the bond's yield and the yield on a
government bond of the same maturity. Measured in bps \\
Ticker Parent & Identifier for the parent company \\
\textbf{Enterprise Value to Sales} & Measures how expensive a company is
relative to its revenue \\
Enterprise Value to PPE & Measures valuation relative to tangible
assets \\
\textbf{Price to Sales} & How much investors pay for \$1 of revenue \\
\textbf{Price to Book Value} & Measures how expensive the firm is
relative to accounting equity \\
Dividend Yield & Return shareholders receive from dividends \\
\textbf{Return on Average Total Equity} & How efficient the firm is with
investments \\
\textbf{Operating Margin} & Core profibility measure \\
Total Capital Expenses to Total Assets & How much firm invests in
long-term assets \\
Book Value per Share & Accounting value per share \\
\textbf{Asset Turnover} & Measures how well assets generate revenue \\
\textbf{Current Ratio} & Measures short-term liquidity \\
\textbf{Cash Dividend Coverage Ratio} & Measures if firm can pay
dividends purely from operations \\
\textbf{Total Debt to Equity} & Measures financial leverage \\
\textbf{Total Debt to Total Capital} & Shows how much of capital is from
debt \\
Amount Outstanding & Size of the bond \\
Years to Maturity & Number of years until the bond pays back its
principal \\
Coupon & Annual payments made to bond holders \\
\end{longtable}

\subsection{Statement of hypothesis/research
question}\label{statement-of-hypothesisresearch-question}

The research question this paper will seek to answer is what are the
micro-determinants of credit spread. We propose that total debt to
equity, price to book value, return on average total equity, and current
ratioare the micro-determinants affect the change in bond spreads over
time.

\section{Literature Review}\label{sec-lit}

Total Debt to Equity + Total Debt to Total Capital
(\textbf{fu2021credit?}) ``we assess the existence of a firm leverage
effect'' ``firm leverage appears to have a significant influence''

(\textbf{wang2023liquidity?}) ROE ``Return on net assets\ldots{} the
higher the ROE, the better the operating conditions and the ability to
service debt.'' higher -\textgreater{} lower spreads Current Ratio
````The higher the current ratio, the better the ability to service
debt\ldots the smaller the credit spread.''\,''

(\textbf{kaviani2020policy?}) Operating Margin ````\ldots operating
income-to-sales ratio\ldots{}''

(\textbf{carvalho2023loan?}) Price to book value (Market-to-book)
````Borrower-level controls include Equity Volatility, Size, Firm Age,
Profitability, Tangibility, Market-to-Book, Leverage, Rated.''\,''
amount outstanding (loan amount) ``I define the net notional amount of
credit risk outstanding for a given firm as: NOf,t is analogous to the
face value of debt outstanding in bond markets---it captures the net
amount of protection sold\ldots{}'' Years to maturity (loan maturity)
``\ldots a firm is defined as a combination of the underlying firm
(e.g.~Ford) and a maturity bucket (e.g.~0-2 years)\ldots{} The maturity
buckets I consider are (in years): 0-2, 2-4, 4-6, 6-8, 8-10, and 10+.''

(Collin-Dufresn, Goldstein, and Martin 2001) added total debt to equity
and enterprice valuue to sales.

\section{Descriptive}\label{sec-des}

\subsection{Tables}\label{tables}

\subsection{Plots}\label{plots}

\includegraphics[width=0.5\linewidth,height=\textheight,keepaspectratio]{spreadvscurrentratio.png}
\includegraphics[width=0.5\linewidth,height=\textheight,keepaspectratio]{spreadvspricetobookvalue.png}

\includegraphics[width=0.5\linewidth,height=\textheight,keepaspectratio]{spreadvsretrnavgtotalequitycubed.png}
\includegraphics[width=0.5\linewidth,height=\textheight,keepaspectratio]{spreadvslogtotaldebttoequity.png}

\subsection{Summary of key variables}\label{summary-of-key-variables}

Spread is numeric with a range from {[}-196.8, 395.8{]}. It had a mean
of -14.6 and a standard deviation of 57.6. It had skweness of 1.017

PricetoBook Value is numeric with a range from {[}-10.3, 7.1{]}. It had
a mean of 0.214 and a standard deviation of 1.17. It had skewness of
0.106.

Return on Average total equity was numeric with a range {[}-90.96,
135.03{]}. It had a mean of -0.47 and a standard deviation of 19.466. It
had a skweness of 1.236.

Current Ratio is numeric with a range from {[}-1.47, 3.27{]}. It had a
mean of -0.14 and a standard deviation of 0.395. It had a skewness of
2.86.

Total Debt to Equity is numeric with range from {[}-80.36, 452.69{]}. It
had a mean of 30.07 and a standard deviation of 39.65. It had a skewness
of 39.65.

\section{Methodology}\label{sec-met}

\subsection{Clear statement of the model and its
assumptions}\label{clear-statement-of-the-model-and-its-assumptions}

\[
y= \beta_{0}+\beta_{1}log(x_1)+\beta_{2}x_2 + \beta_{3}log(x_3)+\beta_{4}x_4
\]

Where \(y\) is the spread of bonds, \(x_1\) is TotalDebttoEquity,
\(x_2\) is PricetoBookValue, \(x_3\) to ReturnonAvgTotalEquity, and
\(x_4\) is CurrentRatio.

\textbf{Linearity in Parameters}

From the model above, it is clear that the model is linear in
parameters.

\textbf{Random Sampling}

This is an assumption about the data given. It is randomly sampled.

\textbf{No Perfect Collinearity}

We ran correlation between all the variables and got the following
table. Since none of the columns are a repeat, there is no perfect
collinearity. \begin{center}
\pandocbounded{\includegraphics[keepaspectratio]{collinearity.png.png}}
\end{center}

\begin{center}
\pandocbounded{\includegraphics[keepaspectratio]{collinearityy.png}}
\end{center}
\textbf{Zero Conditional Mean}

This is an assumption about the population.

\textbf{Homoskedasticty}

We performed robust on all our entries and did not assume
homoskedasticity.

Thus, following all these assumptions, the model is consistent and
unbiased

\begin{center}
\pandocbounded{\includegraphics[keepaspectratio]{twologstwolinear.png}}
\end{center}

\subsection{Specification tests}\label{specification-tests}

Before Adjusting for non linearity: F(12,809) value for estat ovtest,
rhs ( linearity test) = 6.22 with p value 0.0000

After adjusting for non linearity (the model with logs and cubes) :
estat ovtest, rhs F(11, 559) value is 0.89 with p-value 0.5485

The estat ovtest (test for omitted variable bias) only run after
adjusting for non linearity gave F(3,55) value 1.52 with p-value 0.2207

\subsection{Robustness considerations}\label{robustness-considerations}

For robustness considerations, we used robust every time we used
regression. Thus, we never assumed homoskedasticity.

\section{Hypothesis Testing}\label{sec-hyp}

We did not run t-test as we are not siginificance in just one variable
as we started by using these specific variables based on economic
initution.

\begin{center}
\pandocbounded{\includegraphics[keepaspectratio]{jointftest.png}}
\end{center}

\subsection{Choise of appropriate
test}\label{choise-of-appropriate-test}

\section{Results}\label{sec-res}

\subsection{Correct interpretation of
results}\label{correct-interpretation-of-results}

\subsection*{Economic intuition
provided}\label{economic-intuition-provided}
\addcontentsline{toc}{subsection}{Economic intuition provided}

\phantomsection\label{refs}
\begin{CSLReferences}{1}{0}
\bibitem[\citeproctext]{ref-collin2001determinants}
Collin-Dufresn, Pierre, Robert S Goldstein, and J Spencer Martin. 2001.
{``The Determinants of Credit Spread Changes.''} \emph{The Journal of
Finance} 56 (6): 2177--207.

\end{CSLReferences}




\end{document}
