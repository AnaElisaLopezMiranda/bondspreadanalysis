% Options for packages loaded elsewhere
% Options for packages loaded elsewhere
\PassOptionsToPackage{unicode}{hyperref}
\PassOptionsToPackage{hyphens}{url}
\PassOptionsToPackage{dvipsnames,svgnames,x11names}{xcolor}
%
\documentclass[
  letterpaper,
  DIV=11,
  numbers=noendperiod]{scrartcl}
\usepackage{xcolor}
\usepackage{amsmath,amssymb}
\setcounter{secnumdepth}{5}
\usepackage{iftex}
\ifPDFTeX
  \usepackage[T1]{fontenc}
  \usepackage[utf8]{inputenc}
  \usepackage{textcomp} % provide euro and other symbols
\else % if luatex or xetex
  \usepackage{unicode-math} % this also loads fontspec
  \defaultfontfeatures{Scale=MatchLowercase}
  \defaultfontfeatures[\rmfamily]{Ligatures=TeX,Scale=1}
\fi
\usepackage{lmodern}
\ifPDFTeX\else
  % xetex/luatex font selection
\fi
% Use upquote if available, for straight quotes in verbatim environments
\IfFileExists{upquote.sty}{\usepackage{upquote}}{}
\IfFileExists{microtype.sty}{% use microtype if available
  \usepackage[]{microtype}
  \UseMicrotypeSet[protrusion]{basicmath} % disable protrusion for tt fonts
}{}
\makeatletter
\@ifundefined{KOMAClassName}{% if non-KOMA class
  \IfFileExists{parskip.sty}{%
    \usepackage{parskip}
  }{% else
    \setlength{\parindent}{0pt}
    \setlength{\parskip}{6pt plus 2pt minus 1pt}}
}{% if KOMA class
  \KOMAoptions{parskip=half}}
\makeatother
% Make \paragraph and \subparagraph free-standing
\makeatletter
\ifx\paragraph\undefined\else
  \let\oldparagraph\paragraph
  \renewcommand{\paragraph}{
    \@ifstar
      \xxxParagraphStar
      \xxxParagraphNoStar
  }
  \newcommand{\xxxParagraphStar}[1]{\oldparagraph*{#1}\mbox{}}
  \newcommand{\xxxParagraphNoStar}[1]{\oldparagraph{#1}\mbox{}}
\fi
\ifx\subparagraph\undefined\else
  \let\oldsubparagraph\subparagraph
  \renewcommand{\subparagraph}{
    \@ifstar
      \xxxSubParagraphStar
      \xxxSubParagraphNoStar
  }
  \newcommand{\xxxSubParagraphStar}[1]{\oldsubparagraph*{#1}\mbox{}}
  \newcommand{\xxxSubParagraphNoStar}[1]{\oldsubparagraph{#1}\mbox{}}
\fi
\makeatother


\usepackage{longtable,booktabs,array}
\usepackage{calc} % for calculating minipage widths
% Correct order of tables after \paragraph or \subparagraph
\usepackage{etoolbox}
\makeatletter
\patchcmd\longtable{\par}{\if@noskipsec\mbox{}\fi\par}{}{}
\makeatother
% Allow footnotes in longtable head/foot
\IfFileExists{footnotehyper.sty}{\usepackage{footnotehyper}}{\usepackage{footnote}}
\makesavenoteenv{longtable}
\usepackage{graphicx}
\makeatletter
\newsavebox\pandoc@box
\newcommand*\pandocbounded[1]{% scales image to fit in text height/width
  \sbox\pandoc@box{#1}%
  \Gscale@div\@tempa{\textheight}{\dimexpr\ht\pandoc@box+\dp\pandoc@box\relax}%
  \Gscale@div\@tempb{\linewidth}{\wd\pandoc@box}%
  \ifdim\@tempb\p@<\@tempa\p@\let\@tempa\@tempb\fi% select the smaller of both
  \ifdim\@tempa\p@<\p@\scalebox{\@tempa}{\usebox\pandoc@box}%
  \else\usebox{\pandoc@box}%
  \fi%
}
% Set default figure placement to htbp
\def\fps@figure{htbp}
\makeatother


% definitions for citeproc citations
\NewDocumentCommand\citeproctext{}{}
\NewDocumentCommand\citeproc{mm}{%
  \begingroup\def\citeproctext{#2}\cite{#1}\endgroup}
\makeatletter
 % allow citations to break across lines
 \let\@cite@ofmt\@firstofone
 % avoid brackets around text for \cite:
 \def\@biblabel#1{}
 \def\@cite#1#2{{#1\if@tempswa , #2\fi}}
\makeatother
\newlength{\cslhangindent}
\setlength{\cslhangindent}{1.5em}
\newlength{\csllabelwidth}
\setlength{\csllabelwidth}{3em}
\newenvironment{CSLReferences}[2] % #1 hanging-indent, #2 entry-spacing
 {\begin{list}{}{%
  \setlength{\itemindent}{0pt}
  \setlength{\leftmargin}{0pt}
  \setlength{\parsep}{0pt}
  % turn on hanging indent if param 1 is 1
  \ifodd #1
   \setlength{\leftmargin}{\cslhangindent}
   \setlength{\itemindent}{-1\cslhangindent}
  \fi
  % set entry spacing
  \setlength{\itemsep}{#2\baselineskip}}}
 {\end{list}}
\usepackage{calc}
\newcommand{\CSLBlock}[1]{\hfill\break\parbox[t]{\linewidth}{\strut\ignorespaces#1\strut}}
\newcommand{\CSLLeftMargin}[1]{\parbox[t]{\csllabelwidth}{\strut#1\strut}}
\newcommand{\CSLRightInline}[1]{\parbox[t]{\linewidth - \csllabelwidth}{\strut#1\strut}}
\newcommand{\CSLIndent}[1]{\hspace{\cslhangindent}#1}



\setlength{\emergencystretch}{3em} % prevent overfull lines

\providecommand{\tightlist}{%
  \setlength{\itemsep}{0pt}\setlength{\parskip}{0pt}}



 


\KOMAoption{captions}{tableheading}
\makeatletter
\@ifpackageloaded{caption}{}{\usepackage{caption}}
\AtBeginDocument{%
\ifdefined\contentsname
  \renewcommand*\contentsname{Table of contents}
\else
  \newcommand\contentsname{Table of contents}
\fi
\ifdefined\listfigurename
  \renewcommand*\listfigurename{List of Figures}
\else
  \newcommand\listfigurename{List of Figures}
\fi
\ifdefined\listtablename
  \renewcommand*\listtablename{List of Tables}
\else
  \newcommand\listtablename{List of Tables}
\fi
\ifdefined\figurename
  \renewcommand*\figurename{Figure}
\else
  \newcommand\figurename{Figure}
\fi
\ifdefined\tablename
  \renewcommand*\tablename{Table}
\else
  \newcommand\tablename{Table}
\fi
}
\@ifpackageloaded{float}{}{\usepackage{float}}
\floatstyle{ruled}
\@ifundefined{c@chapter}{\newfloat{codelisting}{h}{lop}}{\newfloat{codelisting}{h}{lop}[chapter]}
\floatname{codelisting}{Listing}
\newcommand*\listoflistings{\listof{codelisting}{List of Listings}}
\makeatother
\makeatletter
\makeatother
\makeatletter
\@ifpackageloaded{caption}{}{\usepackage{caption}}
\@ifpackageloaded{subcaption}{}{\usepackage{subcaption}}
\makeatother
\usepackage{bookmark}
\IfFileExists{xurl.sty}{\usepackage{xurl}}{} % add URL line breaks if available
\urlstyle{same}
\hypersetup{
  pdftitle={The Micro-Determinants of Credit Spreads},
  pdfauthor={Nipun Jaiswal and Ana Elisa Lopez-Miranda; (1008726826) and (1008819879)},
  colorlinks=true,
  linkcolor={blue},
  filecolor={Maroon},
  citecolor={Blue},
  urlcolor={Blue},
  pdfcreator={LaTeX via pandoc}}


\title{The Micro-Determinants of Credit Spreads}
\usepackage{etoolbox}
\makeatletter
\providecommand{\subtitle}[1]{% add subtitle to \maketitle
  \apptocmd{\@title}{\par {\large #1 \par}}{}{}
}
\makeatother
\subtitle{ECO375: Empirical Project}
\author{Nipun Jaiswal and Ana Elisa Lopez-Miranda \and (1008726826) and
(1008819879)}
\date{November 27, 2025}
\begin{document}
\maketitle


\newpage

\section{Introduction}\label{sec-intro}

Finding the micro-determinants for credit bonds is useful for finding
what affects yields over time. The yield that investors require to
purchase a firm's bond is the spread over the yield provided by a
comparable government bond. Micro-determinants are variables taht are
related to the probability of default and the liquidity associated with
a specific bond issue. This project seaks to assess the importance of
firm specific factors in determining the spread for its bonds. The data
used covered two points in time: 2022 and 2024. The model is a
cross-sectional problem focusing on the changes in spread from 2022 to
2024. Fixed effects were not included as these were not vary across
bonds.

This paper is structured as follows: Section~\ref{sec-lit} is the
literature review situating the research question, Section~\ref{sec-des}
contains the preliminary analyses, Section~\ref{sec-met} contains the
methodology, Section~\ref{sec-res} contains the results, and
Section~\ref{sec-hyp} has the hypothesis testing.

difference in percentages: basis points

\subsection{Description of variables}\label{description-of-variables}

The variables are as follows

\begin{longtable}[]{@{}
  >{\raggedright\arraybackslash}p{(\linewidth - 2\tabcolsep) * \real{0.4091}}
  >{\raggedright\arraybackslash}p{(\linewidth - 2\tabcolsep) * \real{0.5909}}@{}}
\toprule\noalign{}
\begin{minipage}[b]{\linewidth}\raggedright
Variable
\end{minipage} & \begin{minipage}[b]{\linewidth}\raggedright
Description
\end{minipage} \\
\midrule\noalign{}
\endhead
\bottomrule\noalign{}
\endlastfoot
Issuer & Name of the company that issued the bond \\
Spread & The difference between the bond's yield and the yield on a
government bond of the same maturity. Measured in bps \\
Ticker Parent & Identifier for the parent company \\
Enterprise Value to Sales & Measures how expensive a company is relative
to its revenue \\
Enterprise Value to PPE & Measures valuation relative to tangible
assets \\
Price to Sales & How much investors pay for \$1 of revenue \\
Price to Book Value & Measures how expensive the firm is relative to
accounting equity \\
Dividend Yield & Return shareholders receive from dividends \\
Return on Average Total Equity & How efficient the firm is with
investments \\
Operating Margin & Core profibility measure \\
Total Capital Expenses to Total Assets & How much firm invests in
long-term assets \\
Book Value per Share & Accounting value per share \\
Asset Turnover & Measures how well assets generate revenue \\
Current Ratio & Measures short-term liquidity \\
Cash Dividend Coverage Ratio & Measures if firm can pay dividends purely
from operations \\
Total Debt to Equity & Measures financial leverage \\
Total Debt to Total Capital & Shows how much of capital is from debt \\
Amount Outstanding & Size of the bond \\
Years to Maturity & Number of years until the bond pays back its
principal \\
Coupon & Annual payments made to bond holders \\
\end{longtable}

Enterprise value to sales measures how expensive a company is relative
to its revenue. Thus, a high ratio means the market expects high growth
(or the company is overvalued). Thus, a high enterprise value to sales
means lower perceived risk and lower spreads.

PPE is tangible long-term assets. Thus, a high enterprise value to PPE
means the firm is growth-oriented, but that also means has more risk and
higher spreads.

A high Price to Sales signifies higher confidence in the firm by the
general public which indicates lower spreads.

Price to Book Value is how the firm is valued by the general public. A
lower ratio indicates undervaluation which produces financial distress
and higher spreads.

Divident yield is how much bond holders are receiving from the
dividents. A higher ration may indicate higher risk.

A higher Return on Average total equity indicates the firm is more
efficient which means lower spreads.

The operating margin shows how much earnings are received before
interest and taxes. A higher margin indicates the firm has a lower risk
of going default which means lower spreads.

A high total capital expenses to total assets could indicate either
lower spreads or a lack of liquidy.

A low book value per share may indicate the firm is close to going
default and thus have a higher spread. A high asset turnover indicates
the efficiency which implies lower spreads

A low current ratio indicates liquidity risk which implies higher
spreads.

A low cash dividend coverage ratio indicates weak cash flow which
implies higher spreads

A high total debt to equity means the firm has a high reliance on debt
which implies higher spreads

\subsection{Statement of hypothesis/research
question}\label{statement-of-hypothesisresearch-question}

The research question this paper will seek to answer is what are the
micro-determinants of credit spread. We propose that total debt to
equity, price to book value, return on average total equity, and current
ratioare the micro-determinants affect the change in bond spreads over
time.

\section{Literature Review}\label{sec-lit}

In 2020, Fu et. al employed a multi-factor analysis from both a
firm-specific and market-specific perspective to examine the
determinants of credit spreads in the USA, the UK, and Japan between
2005 and 2012. Their results indicated that the degree of firm leverage
has a siginificant influence on spreads. They mention re-testing the
influence of leverage on spreads, ``as done in earlier work of
Collin-Dufresne e''

Total Debt to Equity + Total Debt to Total Capital Fu, Li, and Molyneux
(2021) ``we assess the existence of a firm leverage effect'' ``firm
leverage appears to have a significant influence''

Wang (2023) ROE ``Return on net assets\ldots{} the higher the ROE, the
better the operating conditions and the ability to service debt.''
higher -\textgreater{} lower spreads Current Ratio ````The higher the
current ratio, the better the ability to service debt\ldots the smaller
the credit spread.''\,''

Kaviani et al. (2020) Operating Margin ````\ldots operating
income-to-sales ratio\ldots{}''

Carvalho, Gao, and Ma (2023) Price to book value (Market-to-book)
````Borrower-level controls include Equity Volatility, Size, Firm Age,
Profitability, Tangibility, Market-to-Book, Leverage, Rated.''\,''
amount outstanding (loan amount) ``I define the net notional amount of
credit risk outstanding for a given firm as: NOf,t is analogous to the
face value of debt outstanding in bond markets---it captures the net
amount of protection sold\ldots{}'' Years to maturity (loan maturity)
``\ldots a firm is defined as a combination of the underlying firm
(e.g.~Ford) and a maturity bucket (e.g.~0-2 years)\ldots{} The maturity
buckets I consider are (in years): 0-2, 2-4, 4-6, 6-8, 8-10, and 10+.''

\section{Descriptive}\label{sec-des}

\subsection{Plots}\label{plots}

\includegraphics[width=0.5\linewidth,height=\textheight,keepaspectratio]{spreadvscurrentratio.png}
\includegraphics[width=0.5\linewidth,height=\textheight,keepaspectratio]{spreadvspricetobookvalue.png}

\includegraphics[width=0.5\linewidth,height=\textheight,keepaspectratio]{spreadvslogreturnavgtotalequity.png}
\includegraphics[width=0.5\linewidth,height=\textheight,keepaspectratio]{spreadvslogtotaldebttoequity.png}

\subsection{Summary of key variables}\label{summary-of-key-variables}

Spread is numeric with a range from {[}-196.8, 395.8{]}. It had a mean
of -14.6 and a standard deviation of 57.6. It had skweness of 1.017

PricetoBook Value is numeric with a range from {[}-10.3, 7.1{]}. It had
a mean of 0.214 and a standard deviation of 1.17. It had skewness of
0.106.

Return on Average total equity was numeric with a range {[}-90.96,
135.03{]}. It had a mean of -0.47 and a standard deviation of 19.466. It
had a skweness of 1.236.

Current Ratio is numeric with a range from {[}-1.47, 3.27{]}. It had a
mean of -0.14 and a standard deviation of 0.395. It had a skewness of
2.86.

Total Debt to Equity is numeric with range from {[}-80.36, 452.69{]}. It
had a mean of 30.07 and a standard deviation of 39.65. It had a skewness
of 39.65.

\section{Methodology}\label{sec-met}

\subsection{Clear statement of the model and its
assumptions}\label{clear-statement-of-the-model-and-its-assumptions}

\[
\Delta y_i= \beta_{0}+\beta_{1}log(x_{i1})+\beta_{2}x_{i2} + \beta_{3}log(x_{i3})+\beta_{4}x_{i4}
\]

Where \(\Delta y\) is the difference in the spread of bonds from 2022 to
2024,

\(x_1\) is the difference in Total Debt to Equity. \(x_2\) is the
difference in Price to Book Value. \(x_3\) is the different in Return on
Average Total Equity. \(x_4\) is the difference in Current Ratio.

\textbf{Linearity in Parameters}

From the model above, it is clear that the model is linear in
parameters.

\textbf{Random Sampling}

This is an assumption about the data given. It is randomly sampled.

\textbf{No Perfect Collinearity}

We ran correlation between all the variables and got the following
table. Since none of the columns are a repeat, there is no perfect
collinearity. \begin{center}
\pandocbounded{\includegraphics[keepaspectratio]{collinearity.png.png}}
\end{center}

\begin{center}
\pandocbounded{\includegraphics[keepaspectratio]{collinearityy.png}}
\end{center}
\textbf{Zero Conditional Mean}

This is an assumption about the population.

\textbf{Homoskedasticty}

We performed robust on all our entries and did not assume
homoskedasticity.

Thus, following all these assumptions, the model is consistent and
unbiased

\begin{center}
\pandocbounded{\includegraphics[keepaspectratio]{twologstwolinear.png}}
\end{center}

\subsection{Specification tests}\label{specification-tests}

Before Adjusting for non linearity: F(12,809) value for estat ovtest,
rhs ( linearity test) = 6.22 with p value 0.0000

After adjusting for non linearity (the model with logs and nooooo cubes)
: estat ovtest, rhs F(12, 46) value is 1.46 with p-value 0.1749

The estat ovtest (test for omitted variable bias) only run after
adjusting for non linearity gave F(3,55) value 1.55 with p-value 0.2129

\subsection{Robustness considerations}\label{robustness-considerations}

For robustness considerations, we used robust every time we used
regression. Thus, we never assumed homoskedasticity.

\section{Hypothesis Testing}\label{sec-hyp}

\begin{center}
\pandocbounded{\includegraphics[keepaspectratio]{jointftest.png}}
\end{center}

\begin{center}
\pandocbounded{\includegraphics[keepaspectratio]{ttest1.png}}
\end{center}

\begin{center}
\pandocbounded{\includegraphics[keepaspectratio]{ttest2.png}}
\end{center}

\subsection{Choise of appropriate
test}\label{choise-of-appropriate-test}

We chose to do a joint f test to see whether any of the parameters
should be 0. We also completed a t test to see individual significance
against the yield.

\section{Results}\label{sec-res}

\subsection{Correct interpretation of
results}\label{correct-interpretation-of-results}

\subsection*{Economic intuition
provided}\label{economic-intuition-provided}
\addcontentsline{toc}{subsection}{Economic intuition provided}

\phantomsection\label{refs}
\begin{CSLReferences}{1}{0}
\bibitem[\citeproctext]{ref-carvalho2023loan}
Carvalho, Daniel, Janet Gao, and Pengfei Ma. 2023. {``Loan Spreads and
Credit Cycles: The Role of Lenders' Personal Economic Experiences.''}
\emph{Journal of Financial Economics} 148 (2): 118--49.

\bibitem[\citeproctext]{ref-fu2021credit}
Fu, Xiaoqing, Matthew C Li, and Philip Molyneux. 2021. {``Credit Default
Swap Spreads: Market Conditions, Firm Performance, and the Impact of the
2007--2009 Financial Crisis.''} \emph{Empirical Economics} 60 (5):
2203--25.

\bibitem[\citeproctext]{ref-kaviani2020policy}
Kaviani, Mahsa S, Lawrence Kryzanowski, Hosein Maleki, and Pavel Savor.
2020. {``Policy Uncertainty and Corporate Credit Spreads.''}
\emph{Journal of Financial Economics} 138 (3): 838--65.

\bibitem[\citeproctext]{ref-wang2023liquidity}
Wang, Haiyang. 2023. {``Liquidity of Corporate Bonds and Credit
Spread.''} \emph{Finance Research Letters} 55: 103941.

\end{CSLReferences}




\end{document}
